注:

1. 模板框架供参考,可根据实际情况调整。

2. 开题报告总体字数不少于20000字,其中,国内外研究现状分析,即文献综述,一般不少于10000中文字,包含至少50篇与本研究方向有关的文献(包括但不限于论文、技术报告、行业标准等)。根据国内外研究现状,应针对本人研究方向或具体问题进行深入调研,对相关研究现状进行系统分析、说明,并指出存在的问题或不足,阐述发展趋势并引出拟研究问题。

3. 格式按照《北京航空航天大学研究生撰写学位论文的规定(2021年11月修订)》。

\textbf{\color{red} 文字论述应具有较强的系统性与逻辑性;文字表达清晰,图表、公式规范}


\section{绪论}

\subsection{论文选题背景}
参考文献引用如下\upcite{yan2018spatial,wu2020comprehensive,hu2015,jdmsugg,lovell1999development}

\textbf{\color{red}(可以从如下几个方面进行论述:1、学术界理论研究背景,2、项目研究背景,3、实际应用背景)}

\subsection{选题意义及必要性}

\clearpage
