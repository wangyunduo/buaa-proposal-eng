\appendix



\section{儿童肌炎评定量表(CMAS-14)} \label{appendix_cmas}

\begin{table}[h]
\zihao{7}
\centering
\caption{儿童肌炎评定量表(CMAS-14)}
\label{cmas_14}
\begin{tabular}{|p{0.5\textwidth}||p{0.5\textwidth}|}
\hline
\textbf{1.抬头(这里指的是平卧位时抬头)} & \textbf{8.举手维持(将手腕举过头顶,并维持)}\\ \cline{1-2}
0=不能 & 0=不能  \\
1=维持1$\sim$9秒  & 1=1$\sim$9秒\\
2=10$\sim$29秒  & 2=10$\sim$29秒  \\
3=30$\sim$59秒  & 3=30$\sim$59秒  \\
4=60$\sim$119秒 & 4=≥60秒\\
5=≥2分钟    &       \\ \cline{2-2}
      & \cellcolor{LightCyan} \textbf{9.坐下(从站立位转成坐在地上)}   \\ \cline{1-2}
\textbf{2.腿/触物(测试者的手放在患儿两只脚的高度)}               & 0=不能   即便允许使用椅子作为帮扶也害怕     \\ \cline{1-1}
0=不能将腿抬离桌面 & 1=非常困难:需要扶椅子才能坐下,如果不扶椅子不愿意尝试 \\
1=可以将腿抬离桌面不能触及测试者的手 & 2=有点困难:坐下时不需要扶椅子,但仍会有点困难,会缓慢小心地坐下,不能完全平衡自己的身体 \\
2=可以将腿抬高至触及测试者的手 & 3=没有困难:没有多余的动作             \\
      &       \\ \cline{1-2}
\textbf{3.伸腿/推持(抬至患儿一只脚的高度)}  & \textbf{10.四肢动作}   \\ \cline{1-2}
0=不能  & 0=俯卧时不能用手和膝关节把身体撑起         \\
1=1$\sim$9秒& 1=可以撑起,但不能保持跪姿,更不能抬头直视前方   \\
2=10$\sim$29秒  & 2=可以保持跪姿,并且能够背部挺直抬头直视前方,但不能向前爬 \\
3=30$\sim$59秒  & 3=可以保持跪姿,并且能抬头向前爬          \\
4=60$\sim$119秒 & 4=可以保持跪姿,并且抬起伸展一条腿时能保持平衡   \\
5=≥2分钟&       \\ \cline{2-2}
      & \cellcolor{LightCyan} \textbf{11.起身(从跪到站)}      \\ \cline{1-2}
\cellcolor{LightCyan} \textbf{4.翻身(仰卧至俯卧)}   & 0=不能   即便允许使用椅子作为帮扶也不行     \\ \cline{1-1}
0=翻身困难,只能轻微或者根本不能将屈曲的右臂压拉到躯干下  & 1=非常困难   需要扶椅子才能站起来        \\
1=翻身尚容易,可以将右臂拉到躯干下方,但不能完全将压在躯干下的右臂拉出,因此不能摆出俯卧姿势 & 2=中等困难   可以不用扶着椅子站起来,但需要按着膝盖、大腿或者地板才能站起         \\
2=翻身很容易,可以完全摆出俯卧姿势,但将右臂从躯干下拉出时有些困难              & 3=轻度困难   不需要协助就可以站起来,但仍有点困难\\
3=轻松翻身,胳膊运动灵活              & 4=没有困难\\
      &       \\ \cline{1-2}
\textbf{5.仰卧起坐(每项完成得1分,共6分)}   & \cellcolor{LightCyan} \textbf{12.从椅上坐起}\\ \cline{1-2}
双手掌紧贴大腿,平衡辅助               & 0=完全不能:即便用手按着椅边也不能坐起     \\
双手臂交叉胸前,平衡辅助               & 1=非常困难:需要用手按着椅边才能坐起      \\
双手握紧置于枕后,平衡辅助              & 2=中等困难   可以不用手按着椅边坐起,但需要用手按着膝或腿才能坐起             \\
双手掌紧贴大腿,无平衡辅助              & 3=轻度困难:不需协助就可以坐起,但仍会有点困难 \\
双手臂交叉胸前,无平衡辅助              & 4=没有困难\\
双手握紧置于枕后,无平衡辅助             &       \\ \cline{2-2}
      & \textbf{13.踏上凳子}          \\ \cline{1-2}
\cellcolor{LightCyan} \textbf{6.坐起(仰卧到端坐)}   & 0=不能  \\ \cline{1-1}
0=不能独立坐起  & 1=非常困难   需要用手扶着测试桌/测试者的手才能踏上   \\
1=相当困雄,非常缓慢费力,几乎不能坐起       & 2=有点困难   可以不用手扶测试桌/测试者的手就能踏上,但需要手按着膝或腿才能踏上      \\
2=有点困难,能够坐起,但是有点缓慢费力       & 3=不需要协助就能完成                \\
3=没有困难&       \\ \cline{2-2}
      & \cellcolor{LightCyan} \textbf{14.拾物}            \\ \cline{1-2}
\textbf{7.举起/伸直手臂}     & 0=不能弯腰捡起地上的铅笔    \\ \cline{1-1}
0=不能将手腕举至肩锁关节平面            & 1=能但非常困难   很大程度上依赖于膝盖和大腿的支撑\\
1=可以举至肩锁关节平面,但低于头顶         & 2=能但有些困雄   至少得扶着膝或腿才能检起,并且动作有些慢\\
2=可以举过头顶,但不能将肘关节完全伸直       & 3=不需要协助就能完成                \\
3=可以举过头顶,并能将肘关节完全伸直        &       \\ \hline
\end{tabular}
\end{table}
